% !TEX program = xelatex

\documentclass[12pt, a4paper]{report}
\usepackage[top = 20mm, bottom = 18mm, left=15mm, right = 15mm]{geometry}
\usepackage{fancyhdr}
\pagestyle{fancy}

\renewcommand{\chaptermark}[1]{%
	\markboth{\chaptername
		\ \thechapter.\ #1}{}}

\usepackage{titlesec, blindtext, color}
\definecolor{gray75}{gray}{0.75}
\newcommand{\hsp}{\hspace{10pt}}
\titleformat{\chapter}[hang]{\LARGE\bfseries}{\thechapter\hsp\textcolor{gray}{|}\hsp}{0pt}{\LARGE\bfseries}
\titlespacing*{\chapter}{0pt}{-30pt}{20pt}

\usepackage{natbib}

\usepackage{amsmath, amsfonts, amssymb, amsthm}
\usepackage[all, cmtip, 2cell]{xy}
\setcounter{tocdepth}{3}
\usepackage{graphicx}
\usepackage{physics}

\usepackage{tikz}

\usepackage{mathtools}
\usepackage{xspace}
\usepackage{enumitem}

\usepackage{tocloft}
\renewcommand{\cftdot}{}

\usepackage{hyperref}
\hypersetup{colorlinks, linkcolor = [RGB]{66, 128, 128}, urlcolor = red, linktocpage = true}
\usepackage{bookmark}
\bookmarksetup{color = [RGB]{66, 128, 128}}

\usepackage[intoc]{nomencl}
\makenomenclature

\usepackage[toc, page]{appendix}

\usepackage[T1]{fontenc}
\usepackage[sfdefault, lining]{FiraSans}
\usepackage{kmath}

\newtheorem{Theorem}{Theorem}[chapter]
\newtheorem{Lemma}[Theorem]{Lemma}
\newtheorem{Corollary}[Theorem]{Corollary}

\theoremstyle{definition}
\newtheorem{Definition}[Theorem]{Definition}
\newtheorem*{Definition*}{Definition}
\newtheorem{Example}[Theorem]{Example}
\newtheorem*{Example*}{Example}
\newtheorem{Exercise}{Exercise}[section]

\theoremstyle{remark}
\newtheorem*{Remark*}{Remark}
\newtheorem*{Solution*}{Solution}
\newtheorem*{Note*}{Note}

\newcommand{\cat}[1]{\mathsf{#1}}
\newcommand{\opp}{^\mathsf{op}}
\DeclareMathOperator{\Ob}{Ob}
\DeclareMathOperator{\Ar}{Ar}
\DeclareMathOperator{\Hom}{Hom}
\newcommand{\fun}[1]{\mathcal{#1}}

\let\Im\relax
\DeclareMathOperator{\Im}{Im}
\newcommand{\id}{\mathrm{id}}
\renewcommand{\th}{\textsuperscript{th}\xspace}

\newcommand{\Mod}[1]{\ (\mathrm{mod}\ #1)}
\DeclareMathOperator{\lcm}{lcm}

%\setlength{\parindent}{0pt}

\begin{document}

\title{\scshape \bfseries Small Categories}
\author{VM}
\date{}
\maketitle

\begingroup
\let\clearpage\relax
\tableofcontents
\endgroup

%\clearpage

\renewcommand{\nomname}{List of Symbols}
\nomenclature[06]{$\mathbb N$, $\mathbb Z$, $\mathbb Q$, $\mathbb R$, $\mathbb C$}{Sets of natural numbers, integers, rational numbers, real numbers, complex numbers (respectively)}
\nomenclature[07]{$\mathbb N_0$}{Set of non-negative integers $\{0, 1, 2, \ldots\}$}
\nomenclature[08]{$\mathbb R_{>0}$}{Set of positive real numbers}

\printnomenclature[10em]

\clearpage

%\include{Introduction}
\chapter{Defining Small Categories}\label{chap:DefSmallCats}

\begin{Example}\label{ex:Mat}
The category $\cat{Mat}_{R}$ has the set of natural numbers $\mathbb N$ as its set of objects and the set of all matrices with entries from the ring $R$ as its set of arrows. For $m, n \in \mathbb N$, the arrows from $n$ to $m$ are all the $m \times n$ $R$-matrices, and composition of arrows is matrix multiplication. Given $A \colon n \to m$ and $B \colon m \to p$, the product $BA$ is defined and is a $p \times n$ matrix --- i.e., $BA$ is an arrow from $n$ to $p$. The identity arrow of $n$ is the identity matrix $I_n$ of order $n$.
\end{Example}
\chapter{Functors}\label{chap:Functors}
\begin{Example}\label{ex:MatFuncs}
Let $r \in \mathbb N$. In the category $\cat{Mat}_R$ given in Example~\ref{ex:Mat}, we can define functors $r \otimes - \colon \cat{Mat}_R \to \cat{Mat}_R$ and $- \otimes r \colon \cat{Mat}_R \to \cat{Mat}_R$ as follows. For any object $n$ of $\cat{Mat}$, $(r \otimes -) n = rn$. For any arrow $A \colon n \to m$, $(r \otimes -) A = I_r \otimes A$, the Kronecker product of the identity matrix $I_r$ and $A$. The other functor $- \otimes r$ is defined similarly.

The Kronecker product of matrices satisfies the property $(A \otimes B)(C \otimes D) = AC \otimes BD$ (provided $A$ and $C$, and similarly $B$ and $D$ are compatible for multiplication). This implies that the functors preserve composition of morphisms. And since $I_r \otimes I_n = I_n \otimes I_r = I_{rn}$, they preserve identity arrows as well.
\end{Example}
\chapter{Adjunctions}\label{chap:Adjunctions}
\begin{Example}\label{ex:MatAdj}
Consider the functor $r \otimes - \colon \cat{Mat}_R \to \cat{Mat}_R$ defined in Example~\ref{ex:MatFuncs}. Let $\fun L = r \otimes -$ and let $\fun R$ be a right adjoint of $\fun L$ (if one exists). Then $\Hom(\fun L(m), n) \cong \Hom(m, \fun R(n))$ for all $m, n \in \mathbb N$. But $L(m) = rm$, so $\Hom(\fun L(m), n)$ is the set of all $n \times rm$ matrices with entries from $R$. If $R$ is finite (e.g., $R = \mathbb Z_2$), then $|\Hom(rm, n)| = |R|^{rmn}$. This suggests that $\Hom(m, \fun R (n)) = \Hom(m, rn)$, so that $\fun R(n) = rn$. This further suggests that $R = r \otimes -$ or $R = - \otimes r$.

Let $R = r \otimes -$. Given an arrow $A \colon \fun L(m) \to n$, that is, given an $n \times rm$ matrix $A$, we must define an arrow $A^* \colon m \to \fun R(n)$, that is, an $rn \times m$ matrix $A^*$, so that $(\cdot)^*$ is a bijection from $\Hom(\fun L(m), n)$ to $\Hom(m, \fun R(n))$. Thus, given an $rn \times m$ matrix $B$, there must be an $n \times rm$ matrix $B^*$, such that $(A^*)^* = A$ and $(B^*)^* = B$ for all $A \in \Hom(\fun L(m), n)$ and $B \in \Hom(m, \fun R(n))$.

Comparing the orders of $A$ and $A^*$ indicates that the operation we need is some sort of matrix transposition --- a ``partial'' one. Consider a partition of the $n \times rm$ matrix $A$.
\begin{equation*}
A = \begin{bmatrix}
A_{11}	&	A_{12}	&	\cdots	&	A_{1m}	\\
A_{21}	&	A_{22}	&	\cdots	&	A_{2m}	\\
\vdots	&	\vdots	&	\ddots	&	\vdots	\\
A_{n1}	&	A_{32}	&	\cdots	&	A_{nm}
\end{bmatrix}_{n \times rm}
\end{equation*}
where each $A_{ij}$ is a $1 \times r$ (row) vector. Then we can obtain a matrix $A^*$ of the required order $rn \times m$ by transposing each such block $A_{ij}$ of $A$.
\begin{equation*}
A^* = \begin{bmatrix}
A^T_{11}	&	A^T_{12}	&	\cdots	&	A^T_{1m}	\\
A^T_{21}	&	A^T_{22}	&	\cdots	&	A^T_{2m}	\\
\vdots	&	\vdots	&	\ddots	&	\vdots	\\
A^T_{n1}	&	A^T_{32}	&	\cdots	&	A^T_{nm}
\end{bmatrix}_{rn \times m}
\end{equation*}
Obviously then, given a matrix $B$ of order $rn \times m$, we can partition it similarly into $n$ rows and $m$ columns of $r \times 1$ column vectors, and then transpose each of these vectors to obtain an $n \times rm$ matrix $B^*$. Both of these operations are well defined, and clearly, $A^{**} = A$ and $B^{**} = B$.

But in order for this bijection to be the natural one between the hom-sets, we must also verify that the following two equations hold for all $\xymatrix{\fun L(m) \ar[r]^{A} & n \ar[r]^C & p}$ and $\xymatrix{q \ar[r]^D & m \ar[r]^B & \fun R(n)}$.
\begin{align*}
\pqty{ \xymatrix{\fun L(m) \ar[r]^{A} & n \ar[r]^C & p} }^* & = \xymatrix{m \ar[r]^{A^*} & \fun R(n) \ar[r]^{\fun R(C)} & \fun R(p)}\\
\pqty{ \xymatrix{q \ar[r]^D & m \ar[r]^B & \fun R(n)} }^* & = \xymatrix{\fun L(q) \ar[r]^{\fun L(D)} & \fun L(m) \ar[r]^{B^*} & n}
\end{align*}
\end{Example}

\end{document}